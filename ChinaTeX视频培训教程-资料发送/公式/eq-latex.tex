\documentclass[t]{beamer}
%\documentclass[handout, t]{beamer}
\setbeamertemplate{navigation symbols}{}

\usepackage[UTF8]{ctex}
\usetheme{AnnArbor}
\usefonttheme{serif}
\useinnertheme{rounded}
\usecolortheme{beaver}
\setbeamertemplate{blocks}[rounded][shadow=true]
\usepackage{graphicx}
 \usepackage{multicol} 
\usepackage{mathrsfs}
\usepackage{bm}
\usepackage{amsmath}
%\usepackage{amsfonts}

\setmainfont{Times New Roman}
\setCJKmainfont{Microsoft YaHei}


\hypersetup{pdfpagemode=FullScreen}


\setbeamercolor{block title}{bg=red!40!white}
\setbeamercolor{block body}{bg=gray!20!white}


%beamerugthemeAnnArbor

\begin{document}

\fontsize{14pt}{.9\baselineskip}\selectfont

  \title{\LaTeX{}公式输入
}
\date{主讲:Jamesfang
}
\author{China\TeX{}在线培训课程}

  \begin{frame}
    \maketitle
  \end{frame}

\begin{frame}{演讲内容}
\begin{multicols}{2}
\tableofcontents
\end{multicols}
\end{frame}

\section{公式基本输入}

\subsection{行内、行间公式}
\begin{frame}[fragile]{行内、行间公式}
\begin{itemize}
\item 行内公式:\verb"$...$"
\item 行间公式:\verb"$$...$$" 或者 \verb"\[...\]"
\end{itemize}

The quick brown fox $\frac{-b\pm\sqrt{b^2-4ac}}{2a}$  jumps over the lazy dog.

The quick brown fox $$\frac{-b\pm\sqrt{b^2-4ac}}{2a}$$  jumps over the lazy dog.
\end{frame}

\subsection{公式的编号}
\begin{frame}[fragile]{公式的编号方法}
\begin{itemize}
\item 自动编号:
\begin{verbatim}
\begin{equation}
...
\end{equation}
\end{verbatim}
\item 标签:\verb"\tag"
\begin{verbatim}
\begin{equation}\tag{...}
   eq.
\end{equation}
\end{verbatim}
\end{itemize}
\end{frame}

\begin{frame}[fragile]
\begin{itemize}
\item 以节为依据进行编号:\verb"\numberwithin{equation}{section}"
\item 子编号:
\begin{verbatim}
\begin{subequations}
\begin{equation}
    ...        (eq. a)
\end{equation}
\begin{equation}
    ...        (eq. b)
\end{equation}
\end{subequations}
\end{verbatim}


\end{itemize}

\end{frame}




\section{常见公式宏包}

\subsection{常见宏包}
\begin{frame}{常见宏包}
\begin{itemize}
\item \texttt{amsmath}宏包
\item 字体宏包
\begin{itemize}\large
\item \texttt{mathrsfs}和\texttt{amsfonts}宏包
\item \texttt{bm}宏包:字体加粗
\end{itemize}
\end{itemize}


\end{frame}

\subsection{字体宏包}
\begin{frame}[fragile]{\texttt{mathrsfs}和\texttt{amsfonts}宏包}
\begin{itemize}
\item \verb"\mathscr" $$\mathscr{ABCDEFGHIJKLMNOPQRST}$$
\item \verb"\mathcal"$$\mathcal{ABCDEFGHIJKLMNOPQRST}$$
\item \verb"\mathbb"$$\mathbb{ABCDEFGHIJKLMNOPQRST}$$
\item \verb"\mathfrak"$$\mathfrak{ABCDEFGHIJKLMNOPQRST}$$
\end{itemize}
\end{frame}

\begin{frame}[fragile]{\texttt{bm}宏包}
可用于字体的加粗
$$\bm{x, \; X,\; \alpha, \; \Theta};\qquad x, \; X,\; \alpha, \; \Theta$$
$$\bm{x^2+y^2=z^2};\qquad x^2+y^2=z^2$$

另一种形式的粗体:使用\verb"\mathbf"命令
$$\mathbf{x,y}; \quad \mathbf{X,Y}$$
\end{frame}


\subsection{公式的环境}
\begin{frame}{公式的环境}
\begin{itemize}
\item 矩阵环境:\texttt{array}, \texttt{matrix}, \texttt{Bmatrix}, \texttt{bmatrix}, \texttt{pmatrix},   \texttt{vmatrix},  \texttt{Vmatrix},
\item 分段函数环境:\texttt{cases}
\item 公式对齐环境:\texttt{split}, \texttt{align},  \texttt{eqnarray}, \texttt{gathered}
\end{itemize}
\end{frame}


\section{定理环境}
\subsection{定理宏包}
\begin{frame}{定理宏包}
\begin{itemize}
\item \texttt{amsthm}宏包
\item \texttt{ntheorem}宏包

\end{itemize}
\end{frame}



\subsection{定理环境的设置}
\begin{frame}[fragile]{定理环境的设置}
\begin{verbatim}
\newtheorem{thm}{定理}[chapter]
\newtheorem{defn}{定义}
\newtheorem{lemma}[thm]{引理}
\end{verbatim}
\end{frame}


\subsection{定理的样式}
\begin{frame}[fragile]{定理的样式}
\begin{verbatim}
\theoremstyle{plain}
\theoremstyle{definition}
\theoremstyle{remark}
\end{verbatim}

\end{frame}





\subsection{证明环境}
\begin{frame}[fragile]{证明环境}
\verb"proof"环境

证明结尾符号\verb"\qed, \qedhere"
\end{frame}


\begin{frame}
\begin{center}\Huge
{\color{red}谢谢观赏!}\\
JamesFang\\
Administrator of {\color{blue}China\TeX{}.org}
\end{center}
\end{frame}









\end{document}
